\documentclass{report}
\usepackage[utf8]{inputenc}
\usepackage[fontsize=15pt]{fontsize}
\title{\textbf{ME5010 : Logistic Map}}


\begin{document}

\maketitle

\section{Introduction}

\begin{center}$x_{n+1} = rx_n(1-x_n)$\end{center}
Logistic Equation is primarily know for modeling population growth of animals and it is part \textbf{Chaos Theory}, which is branch of mathematics that demonstrated how deterministic mathematical system could lead to unpredictability.

Due to certain characteristic of logistic map it was once used to generate pseudo-random numbers. It could generate unpredictability from deterministic machine.

Those characteristic of logistic map was it's sensitivity to input parameters. Very small changes in $x_0$ and $r$ could lead to drastic changes in predictions, and thus leading to Chaos. Due to this particular nature logistic map or similar equation could  be used in cryptography for encryption of data where encrypted data would be very sensitive to the key i.e small variation in key would not decrypt/decode the encrypted data allowing only one and one unique key to decrypt the data.
\end{document}
